\documentclass[a4paper,norsk,11pt]{article}
\usepackage[utf8x]{inputenc}
\usepackage{babel}
\usepackage[firstpage]{draftwatermark}

%opening
\title{Bruksanvisning for Sonja}
\author{}\date{}
\SetWatermarkColor{red}
\SetWatermarkText{Dette dokumentet er under arbeid!}
\SetWatermarkLightness{0.7}
\begin{document}

\maketitle

\section{Noen tips for å komme i gang med testinga}

\subsection*{Innledning}
\begin{enumerate}
\item sjekk at programmet har oppførsel som stemmer overens med dine forventninger.
\item forsøk å gjøre endringer/rettelser som du mener er feil og som du forventer at programmet skal reagere på
 \item gjør endringer i originale data og lagre dem. Start programmet på nytt med oppdaterte data og sjekk at rettelser og tilføyelser er med.
\end{enumerate}

\begin{description}
 \item[\textbf{Datagrunnlag}] Det første du blir bedt om, er hvilket datasett du vil bruke. Du kan velge mellom to: et originalt som er identisk med det som i øyeblikket brukes i websøket (Test - originaldata); det andre er datasettet lagret av deg etter at du har gjort rettelser og tilføyelser (Test - oppdaterte data). De øvrige valgene skal du ikke bry deg om.
\item[\textbf{Søking}] Når dataene er initiert, kan du søke på samme vis som i Roald.
\item[\textbf{Trefflista}] her er det flere muligheter:
\begin{itemize}
 \item et vanlig klikk markerer et treff og viser detaljer om termen eller strengen.
\item et høyreklikk vil sette inn treffet som seogså term til den termen som er valgt med vanlig klikk (gjelder termer)
\item dersom du er i ferd med å bygge en streng vil Ctrl-klikk sette inn valgt term i det feltet du holder på å bygge.
\end{itemize}
\end{description}

\section{Jobbe med termer}
\begin{description}
\item[Registrere ny term] velg funksjon fra Omveier-menyen og skriv inn betegnelsen og velg termtype.
\item[Fjerne term] velg funksjon fra omveier-menyen. Her er det ennå ikke hundre prosent behandling av strenger der termen kunne tenkes å forekomme.
\item[Flette to termer] velg funksjon fra omveier-menyen, oppgi først dem som skal bli stående, deretter den som skal slås inn under første term.
\item[Bytte BF og term] velg funksjon fra omveier-menyen. Viss det er mer enn én BF, kommer det opp en meny som du kan velge fra.
\item[Endre data] klikk i feltene for å legge til eller endre data i dem. Eller bruk knappene ved sidene av feltene
\item[Lagre data] velg Lagre grunnlag fra Eksport
\end{description}

\section{Jobbe med strenger}
Lite ferdig her ennå, f.eks ikke behandling av strengeforslag fra Roald.
\begin{description}
 \item[Fjerne streng] skal virke, sjekk at strengen er fjernet
\item[Bygge ny streng] Knappen endrer navn og du kan begynne å klikke i feltene for å legge inn data. Dersom betegnelsen du oppgir bare har ett treff i termtypen, blir den lagt inn autmatisk. I motsatt fall får du en treffliste de kan CTRL-klikke i for å velge. Sjekk at strengen er kommet inn i første ledd termliste
\end{description}

\end{document}
